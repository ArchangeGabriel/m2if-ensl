	\documentclass[a4paper,10pt]{article}
\usepackage[utf8]{inputenc}
\usepackage{amsfonts}
\usepackage{amsmath}
\usepackage{proof}
\usepackage{amsthm}

\newtheorem{definition}{Definition}
\newtheorem{theorem}{Theorem}
\newtheorem{proposition}{Property}
\newtheorem{lemm}{Lemma}
\newtheorem{corollary}{Corollary}
\newtheorem{ex}{Example}
\newtheorem{exo}{Exercise}
\newtheorem{rmk}{Remark}



\author{Guillaume Hanrot \and Damien Stehlé \\ \and (transcribed by the students)}

\title{Cryptology} 

\begin{document} 

\maketitle

C. Hermite and H. Minkowski initiated the field.

\section{Euclidean Lattices}
\subsection{Definitions}

\begin{definition}
Let $b_1$, $b_2$,...,$b_d$ be linear independent vectors in $\mathbb{R}^n$ ($d \leq n$). The lattice they span is $L(b_1,b_2,...,b_d)= \sum_{i \leq d} x_i b_i$, where $x_i \in \mathbb{Z} \}$
\end{definition}

For example:
\begin{itemize}
\item Trivial lattice $\{0\}$
\item $\mathbb{Z}^2$ is a lattice (for example with $d=2$, $b_1=(0,1)$, $b_2=(1,0)$)
\item $\mathbb{Z}^n$ for any $n$
\end{itemize}

\begin{definition}
A lattice is a discrete additive subgroup of $\mathbb{R}^n$ for some $n$. (Discrete means that there is no accumulation point).
\end{definition}

\begin{theorem}
The two definitions are equivalent.
\end{theorem}

\begin{proof}
\textbf{Exercise}
\end{proof}

For example:
\begin{itemize}
\item Any additivesubgroup of $\mathbb{Z}^n$ is a lattice
\item Let $a_1, a_2 \in \mathbb{Z}$, let $q \neq 0$. $\{ (x_1,x_2) \in \mathbb{Z}^2 \ | \ x_1 a_1+x_2 a_2 = 0 \ [q]  \}$ is a lattice
\item More generally, let $A \in \mathbb{Z}^{m*n}$. $\{ x \in \mathbb{Z}^m \ | \ xA (\in \mathbb{Z}^n)=0 \ [q]$ \textit{(component wise)}$\}$ is a lattice
\item $\mathbb{Z}+\sqrt{2} \mathbb{Z}$ is \textit{not} a lattice. (Not discrete)
\end{itemize}

\begin{definition}
If a lattice $L$ is contained in some $\mathbb{Z}^n$, we call $L$ \textbf{integral}.
\end{definition}

\subsection{Dimension and bases}

\begin{definition}
If $L=L(b_1,b_2,..,b_d)$ with $b_1,....,b_d$ linearly independent, then $(b_1,b_2,...,b_d)$ is called a basis of $L$.
\end{definition}

\begin{proposition}
If $b_1,b_2,...b_d$ and $c_1,...,c_{d'}$ are two bases of the same lattice $L$, then $d=d'$. We call $d$ the \textbf{rank} (or \textbf{dimension}) of the lattice. $n$ is the \textbf{embedded dimension}. If $d=n$, we call the lattice \textbf{full-rank}.
\end{proposition}

\begin{proof}
$d=dim(Span(b_1,b_2,...b_d))=dim(Span(c_1,c_2,...c_{d'}))=d'$
\end{proof}

\begin{definition}
$U \in \mathbb{Z}^{d*d}$ is called \textbf{unimodular} if $det \ U = ^+_-1$
\end{definition}

\begin{proposition}
$U$ unimodular $\Rightarrow$ $U^{-1}$ is unimodular.
\end{proposition}

For example:
\begin{itemize}
\item $U_{i,i}=^+_- 1$ and $U_{i,j}=0$ otherwise
\item Permutation matrices: $U_{i,j}=1$ if $j=\sigma(i)$ and $U_{i,j}=0$ otherwise
\item $U_{i,i}=^+_- 1$ and $U_{i,j}=0$ if $i < j$ (resp. $j<i$)  $U_{i,j}\in \mathbb{Z}$ if $j < i$ (resp. $i<j$)
\end{itemize}

A lattice is most often represented by one of its bases.
A basis is usually encoded into an $n*d$ matrix.
$L$ is the set of integer linear combinations of the columns of the basis matrix (if we use column vectors).
When $2 \leq d$, any given lattice $L$ has infinitely many bases.

\begin{theorem}
Let $b_1, b_2,...,b_d$ and $b_1',...,b_d'$ be two bases of a lattice $L$.
Then there exists $U$ unimodular such that $(b_i')_i . U=(b_i)_i$
Moreover, if there is $U$ unimodular such that $(b_i')_i . U=(b_i)_i$, then $(b_1,b_2,...,b_d)$ and $(b_1',...,b_d')$ span the same lattice.
\end{theorem}

\begin{proof}
\begin{itemize}
\item $b_1' \in L(b_1,b_2,...,b_d)$. So there exists  $u_{1 1},u_{1 2},...u_{1 d} \in \mathbb{Z}$ such that $b_1'= \sum_{j=1}^d u_{1 j} b_j$. Similarily, $b_i'=  \sum_{j=1}^d u_{i j} b_j$ for some $u_{i j}$s in $\mathbb{Z}$.

$B' = B U$ 

Similarily, there exists $U' \in \mathbb{Z}^{d*d}$ such that $B = B' U'$.

Overall, $B= B' U' = B U U'$. So $B(Id-U U')=0$.
Since $B$ has rank $d$, there exists a subset of the rows of $B$, of cardinality $d$, which is square and invertible. Call $C$ this matrix in $\mathbb{R}^{d*d}$. We have $C(Id-U U')=0$. So $U U'=Id$
\item $(b_i')_i U = (b_i)_i$ with $U$ unimodular means that any $b_i$ belongs to $L(b_1',b_2',..,b_d')$. Hence $L(b_1,b_2,..,b_d) \subset L(b_1',b_2',..,b_d')$

Moreover, in $(b_i)_i U^{-1} = (b_i')_i$, $U^{-1}$ is unimodular. So $L(b_1',b_2',..,b_d') \subset L(b_1,b_2,..,b_d)$
\end{itemize}
\end{proof}

To move from one basis to another, you can:
\begin{itemize}
\item Multiply vectors by $^+_- 1$
\item Permutate vectors
\item $b_i \leftarrow b_i + x b_j$ with $x \in \mathbb{Z}$ and $j \neq i$ 
\end{itemize}

\begin{definition}
Let $b_1,...,b_d$ be a basis. We call fundamental parallelepiped of the $b_i$s the set:

$P(b_1,b_2,...,b_d)=\{\sum^d_{i=1} y_i b_i, \ y_i \in [0,1) \}$
\end{definition}

\begin{proposition}
$Span(L(b_1,b_2,...,b_d))=$ (Disjoint union of, for $b \in L$,) $\{ b+P(b_1,b_2,...,b_d) \}$
\end{proposition}

Two algorithmic questions :

\begin{itemize}
\item How do I test that a point belongs to a lattice?
\item How do I test if $L(b_1,b_2,...,b_d)=L(b_1',b_2',...,b_d')$ ? What if the $b_i$s or $b_i'$ are not linearly independent?
\end{itemize}

\subsection{The Hermite Normal Form (HNF)}
Provides a uniquely determined way of representing any \textbf{integral} lattice.

\begin{definition}
$H \in \mathbb{Z}^{n*d}$ with rank $r$ is said to be in HNF if:
\begin{itemize}
\item There exist $1 \leq i_1 < i_2 < ... < i_r \leq n$ such that $H_{ij}=0$ if $i < i_j$
\item $H_{ij} =0$ if $j>r$
\item $0 \leq H_{i_j,j} < H_{i_j,i_j}$ forall $j < i_j$
\end{itemize}
\end{definition}

\begin{theorem}
For any $B \in \mathbb{Z}^{n*d}$, there exists $U \in \mathbb{Z}^{d*d}$ unimodular such that $B U$ is in HNF. Moreover, $U$ can be computed in polynomial time from $B$ ("in polynomial time" meaning here: relatively to $n$, $d$ and $log \ B \ = max \ log(1+|B_{i,j}|)$).
\end{theorem}

\begin{proof}
Micciancio and Warinski, ISSAC'01

We consider the case of $B$ being square and invertible
\begin{lemm}
Assuming that there exists some $H= B U$ in HNF, then $0 \leq H_{i,j} < |det(B)|$
\end{lemm}
\begin{proof}
$|det \ H|=|det \ B|=\pi_{i \leq d} H_{i,i}$ (which belongs in $\mathbb{Z}$)

So $\forall i, \ H_{i,i} \leq |det \ B| $ and $H_{i,i}$ dominates the row.

Consequently, $H_{i,j} < |det \ B|$ for $j < i$
\end{proof}

\begin{lemm}
$|det \ B| \leq \pi_{i \leq d} || b_i || $ (Hadamard's bound) ($||.||$ is the euclidean norm)
\end{lemm}
$|det \ B| \leq \sqrt{d}^{d} \pi || b_i || \leq \sqrt{d}^{d} max |B_{i,j}|^d$
\begin{lemm}
In the extended gcd Euclidean algorithm, the "Bezout matrix" is unimodular
\end{lemm}
\begin{proof}
$(r_0,r_1,...,r_t,r_{t+1})$

Where $r_0$ and $r_1$ are the input, and $r_{t+1}=0$, and $r_t=gcd(r_0,r_1)>0$

$\forall i, we have (r_i,r_{i+1})=(r_i,r_i-q_i r_{i-1})$

The corresponding matrix is unimodular.

The bezout matrix is the product of all those matrices, which enables us to compute $(gcd,0)$ from $(r_0,r_1)$.

It is unimodular as a product of unimodular matrices.
\end{proof}
By induction, assuming $2 \leq d$

We call $b$ the last column of $B$. We call $A$ the $d-1*d-1$ matrix at the top-left corner of $B$. And $^ta$ the last row (deprived of its last coefficient).

Up to a permutation of the columns of $B$, $A$ is invertible (example of square invertible $B$ where $A$ is not invertible without permutation: $b_{i,j}=1$ if $i \leq j$, $0$ otherwise)

Now let $U_A \in \mathbb{Z}^{d-1*d-1}$ and $H_A$ in HNF such that

$A U_A = H_A$

Apply the extended $U_A$ (equals the identity on the new elements of the basis)(it's unimodular) of size $d$ to $B$ and get $B'$, which we represent as $H_A$, $b'$ and $a'$ (according to the previously used representation) \textbf{A drawing would be nice here}
(Complexity-wise, $U_A = A^{-1} H_A$ can be computed in time polynomial in the bit-size of $A$, which is itself polynomial in the bit-size of $B$) \newline

How to clean the first entry of $b'$?
Do extended-Euclid on $r_0 = H_{1,1}$ and $r_1 = b'_1$
$(r_0,r_1).U = (gcd,0)$ where $U$ is of size $2*2$. 
\textbf{Someone draw the matrices!}
Keep going with $(H_{2,2},b'_2)$, $(H_{3,3},b'_3)$, etc and it ends up being lower triangular.
Then use the right substraction of vectors to ensure that $0 \leq H_{i,j} \leq H_{i,i} \ \forall i, j$
(Exercise: Could we do it for upper-triangular HNF?) \newline

How to make this polynomial-time? (How to prevent entries from blowing up?)
Thanks to $L_1$ and $L_2$, we know that the entries of $H_A$ are small.
We are to ensure that the entries of the first ($H_1$) and last ($H_d$) columns are small, after having applied \textbf{drawing!}
For $i$ from $2$ to $d-1$, do
$H_1 \leftarrow H_1 - H_{i,i}*(floor \ of \ H_{1,i}/H_{i,i})$ 
(At the end of this, $\forall i, |H_{1,i}| \leq H_{i,i} \leq |det \ A| \leq 2^{poly(lg \ B, d)}$)
Do the same with $H_d$.
(Exercise : Clean the complexity proof)
\end{proof}

\begin{proof}
And in the general case? What if $B$ is not "square and invertible"?

\begin{itemize}
\item If $B$ has full row rank, $rank(B)=n$. We take a subset of the columns that is of cardinality $n$ and whose elements are linearly independant. (Use permutation matrices). Do HNF on the first $n$ columns, and add columns as we did before.
\item Otherwise, take a subset of rows that are linearly independant, apply the above, and apply the transformation matrix to the full B. 
\end{itemize}
\end{proof}

\begin{theorem}
The HNF is unique
\end{theorem}
\begin{proof}
By contradiction.

Assume that $L(B)=L(B')$ with $B \neq B'$ (both in HNF)
Let $j$ be the minimal integer such that $b_j \neq b'_j$
Let $i$ be the minimal integer such that $b_{i,j} \neq b'_{i,j}$

Look at $b_j-b_j'$ (which is in $L$): its first $i-1$ entries are $0$. 

$b_j-b'_j = \sum_{k \leq d} x_k*b_k = \sum_{k=i}^d x_k b_k$ (can't use the first vectors of HNF)
So $i$ is one of the $i_k$'s and $b_{i,j}-b'_{i,j}=x b_{i_k} i_k$ for some $x \in \mathbb{Z}$.
Impossible because $0 \leq b_{i,j}, b'_{i,j} < b_{i_k,i_k}$ (The diagonal coefficient is the largest one in its row)
\end{proof}

(End of the first course)

\subsection{Lattice invariants}

The lattice rank ($d$) and the embedding dimension are lattice invariants (independant from the choice of the basis)

\begin{definition}
Let $L$ be a lattice $\neq \{ 0 \}$. The first minimum of $L$, denoted by $\lambda_1 (L)$, is the norm of a shortest non-zero vector: $\lambda_1(L)=min\{||b||, b \in L \backslash \{ 0 \}\}$
\end{definition}

Exercise: Prove that in dimension $d$, the minimum is reached no more than $3^d$ times.

\begin{definition}
If $k \leq d$, the $k^{th}$ minimum is $\lambda_k(L)=min\{r | B(0,r)$ contains at least $k$ linearly independant vectors of $L \}$
\end{definition}

\begin{proposition}
If $L$ has rank $d$, then there exist $b_1,...b_d$ in $L$ linearly independant such that: $\forall i, ||b_i||=\lambda_i(L)$
\end{proposition}

\begin{proof}
Left as an exercise.
\end{proof}

Exercise: Find a lattice such that no basis reaches all minima. (Tip: $5 \leq d$)

\begin{definition}
Let $L$ be a lattice with a basis $B=(b_1,...,b_d)$. The determinant/volume of $L$ is defined as $det(L)=\sqrt{det(B^t B)}$, where $B^t B$ is the Gram matrix of the $b_i$'s. 
\end{definition}

Remark: If $d=n$ then $det(L)=|det(B)|$.
$det(L)=$ Area of the fundamental parallepiped $P(b_1,...,b_d)$

\begin{proposition}
$det(L)$ does not depend on the choice of $B$.
\end{proposition}

\begin{proof}
Let $B$, $B'$ be two bases of $L$. $\exists U$ unimodular such that $B'=B U$.
Then $det((B')^t B')=det(U^t (B^t B) U)=det(U^t) det(B^t B) det(U)=(det(U))^2 det(B^t B)=det(B^t B)$
\end{proof}

\subsection{Minkowski's theorem(s)}

Which will link the $\lambda_i$'s and the $det$, respectively hard and easy to compute.

\begin{theorem} (Blichfeld's theorem)
Let $L \subset \mathbb{R}^n$ be a full-rank lattice. Let $S \subset \mathbb{R}^n$ with $vol(S)>det(L)$ ($vol$ in the sens of Lebesgue). Then there exist $z_1 \neq z_2 \in S$ such that $z_1-z_2 \in L$.
\end{theorem}

Exercise: What if we remove the "full-rank" condition?

\begin{proof}
Let $B$ be a basis of $L$.
We have: $\mathbb{R}=$(disjoint union for $b \in L$)$[P(B)+b]$

Let $S_b=S$ \textbf{intersection} $[P(B)+b]$

We have: $S=$ (disjoint union for $b \in L$) $S_b$. Consequently, $vol(S)= \sum_{b \in L} vol(S_b)$

Let $T_b=S_b-b$. We have: $\forall b, T_b \subset P(B)$.

And $\sum vol(T_b)= \sum vol (S_b) = vol(S) > det(L) = vol(P(B))$

There must exist $T_{b_1} \neq T_{b_2}$ such that $T_{b_1}$ \textbf{inter} $T_{b_2} \neq$ \textbf{void} (Lemme des tiroirs)

Take $z \in T_{b_1}$ \textbf{inter} $T_{b_2}$ and define $z_1=z+b_1$ and $z_2=z+b_2$.
\end{proof}

\begin{theorem} (First Minkowski's theorem)
Let $L \subset \mathbb{R}^n$ be a full-rank lattice.

Let $S \subset \mathbb{R}^n$ be convex and symmetric with $vol(S)>2^n det(L)$.

Then $S$ contains a non-null lattice point.

Furthermore, if $S$ is compact, it suffices to have $2^n det(L) \leq vol(S)$.
\end{theorem}

\begin{proof}
\begin{enumerate}
\item Apply Blichfeld to $T=frac{S}{2}$ $(vol(T)>det(L))$. There exists $z_1 \neq z_2$ in $T$ such that $z_1-z_2 \in L$. Let $b=z_1-z_2 \in L \backslash \{ 0 \}$.

$2 z_1 \in S$. 

$2 z_2 \in S$. 

$-2 z_2 \in S$. (symmetry) 

$\frac{2 z_1 - 2 z_2}{2} \in S$. (convexity) 

$z_1-z_2 \in S$.

\item Use first part of the theorem with $S_n = S (1+ \frac{1}{n})$. Build sequence of points and use compacity (converging sub-sequence)(textbf{Left as an exercise})
\end{enumerate}
\end{proof}

\begin{corollary}
If $L$ is a $d$-dimensional lattice, there exists $b \in L \backslash \{ 0 \}$ such that $||b||_{ infinity} \leq det(L)^{1/d}$ ($||b||_{infinity}$ is $|max \ coeff \ of \ b|$)
If $L$ is a $d$-dimensional lattice, there exists $b \in L \backslash \{ 0 \}$ such that $||b||_2 \leq det(L)^{1/d}$

The latter implies $\lambda_1 (L) \leq \sqrt{L}[det(L)]^{1/d}$
\end{corollary}

\begin{proof}
For the first point, use Minkowski's theorem with $S=[-R;R]^n$, with $R=det(L)^{1/d}$.

For the second point, use Minkowski's theorem with $S=B(0,R)$, with $R=det(L)^{1/d}$.

Or: $\forall b, \ ||b||_2 \leq \sqrt{d} ||d||_{infinity}$ ($||b||_2 = \sum b_i^2 \leq d \ max |b_i|^2 = d ||b||_{infinity}^2$)
\end{proof}

\begin{definition}
§Let $2 \leq d$. We call \textbf{Hermite constant} the quantity $\gamma_d = sup_{L \ d-dim \ lattice} [\frac{\lambda_1 (L)}{det(L)^{1/d}}]^2$
\end{definition}

We know this quantity for only very few dimensions ($d \leq 8$ and $d=24$(Cohn, Kumar '04))

\begin{theorem} (Second Minkowski's theorem)
$\pi_{i \leq d} \lambda_i(L) \leq \gamma_d^{\frac{d}{2}} det(L) \leq d^{\frac{d}{2}} det(L)$
\end{theorem}

($\Rightarrow$ First Minkowski's theorem)

\begin{proof}
Left as an exercise (try to derive it from first Minkowski's theorem)
\end{proof}

\section{Lattice reduction, or how to compute short non-null vectors}

\subsection{Some algorithmic problems}

\begin{definition}
\begin{itemize}
\item \textbf{Decisional SVP (Shortest Vector Problem)}: Given a basis of a lattice $L$, and a real $r>0$, say whether $\lambda_1(L) \leq r$ or not.
\item \textbf{Computational SVP}: Given a basis of a lattice $L$, find $b \in L$ such that $||b||=\lambda_1(L)$
\item \textbf{Dec-Gap-SVP$_{\gamma}$} ($\gamma \geq 1$): Given a basis of a lattice $L$ such that $\lambda_2(L) \geq \gamma \lambda_1(L)$ and $r>0$, say whether \textbf{Missing part here!!!!!}
\item \textbf{Computational-SVP$_{\gamma}$} ($\gamma \geq 1$): Given a basis of a lattice $L$, find $b \in L$ such that $||b|| \leq \gamma \lambda_1(L)$
\item \textbf{Dec-CVP$_{\gamma}$ (Closest Vector Problem)}; Given a basis of a lattice $L$, a target $t \in Span(L)$, a real $r>0$ and a promise that either $min_{b \in L} ||b-t|| \leq r$ or $> \gamma^r$. Say in which of the two cases we are. (With $\gamma = 1$, is there a $b \in L$ within distance $r$ of $t$?)
\item \textbf{Comp-CVP$_{\gamma}$}: Given a basis of $L$ and a target $t \in Span(L)$, find $b \in L$ such that $||b-t|| \leq \gamma min_{c \in L} ||c-t||$
\end{itemize}
\end{definition}
\textbf{Here the complexity scale}

The best known algorithms for SVP/CVP:

Kamman ('83): $d^{O(d)}$-time, $d^{O(1)}$-space.

Micciancio-Voulgaris ('10): $2^{O(d)}$-time and space

\begin{definition}
\textbf{$\gamma$-BDD (Bounded Distance Decoding)} ($\gamma \geq 2$): Given a basis of a lattice $L$, and a target vector $t$ such that $\exists b \in L, \ ||b-t|| \leq \frac{\lambda_1(L)}{\gamma}$,

Find b.
[There is a unique solution]
\end{definition}

\subsection{Gram-Schmidt Orthogonalisation}

\begin{definition}
Let $b_1,b_2,...b_d$ be linearly independant.

We define $b_i^* = b_i - \sum_{j < i} \mu_{i,j} b_j^*$ with $\mu_{i,j} = \frac{(b_i,b_j^*)}{||b_j^*||^2}$

(The $b_i$'s are orthogonal, $\forall j, Span_{i \leq j}(b_i^*)=Span_{i \leq j}(b_i)$ (But the $b_i^*$ may not be in $L$))
\end{definition}

\begin{proposition}(Pythagore)
$\forall i, ||b_i||^2 = ||b_i^*||^2+\sum_{j<i} \mu_{i,j}^2 ||b_j^*||^2$
\end{proposition}

Remark: We can write $B= B^* M^t$, where $B^*$ is $[b_1^*,b_2^*,...,b_d^*]$ and $M^t$ is upper-triangular, with all diagonal coefficients at $1$.

$B = B^* diag(||b_i^*||^{-1}) diag(||b-i^*||) M^t$

$B= Q R$

\textbf{À compléter - ce qui suit est le cours 4}

\subsection{The RSA cryptosystem}

Rivest, Shamir, Adleman ('78)

$p, q$ large prime numbers.

Compute $N= P Q$

Pick an integer (usually small)(usually 3, 17, or 65537) $e$ coprime with (p-1)(q-1).

Define $d=e^{-1} mod (p-1)(q-1)$

Then $p_k=(N,e)$ and $s_k=(p,q,s)$.

\textit{Encryption}:

Convert the data into elements of $\mathbb{Z}/ n \mathbb{Z}$, say $m$.
$m \rightarrow m^e mod N$

\textit{Decryption}:

$c \rightarrow c^d mod N$

\textit{Soundness}:
What happens, starting from m, if I encrypt then decrypt?
$m \rightarrow m^e mod N$
$m \rightarrow m^{e d} mod N$

\begin{proposition}
$m^{e d} = m (mod N)$
\end{proposition}

\begin{proof}
Let us look at :
$m^{e d} mod \phi$
$ed = 1 (mod (p-1)(q-1))$
Hence
$ed=1+k(p-1)(q-1)$ for some $k$.

$m^{1+k(p-1)(q-1)}= m ((m^{p-1})^{k(q-1)}) (mod p)$

\begin{itemize}
\item $m \neq 0 (mod p)$
Fermat's little theorem: $m^{p-1}=1 mod p$
$\Rightarrow m^{e d}=m (mod p)$
\item $m=0 (mod p)$
$m^{e d}=0=m (mod p)$
\end{itemize}
\end{proof}

Consequence: In all cases, $m^{e d}= m(mod p)$. Similarly, $m^{e d}=m ()mod q$
$\Rightarrow m^{e d}=m(mod N)$

\textit{Hardness assumptions}

\begin{enumerate}
\item Starting from $N$, hard to recover $p$ and $q$ [Factoring-Record $N = 2^{768}$]
\item Starting from $N$, $e$, hard to recover d [Probabilistic-polynomial time equivalent to (1)] ((1) and (2) are attachs on the key)
\item Starting from $c$, recover $m$ [extract $e^{th}$ roots]
\item CPA: Hard to deduce any information on $m$ given $c$ (Even want that it is hard to decide if $m_1=m_2$ given $c_1$ and $c_2$)(This is BAD for RSA, because it requires the encryption function to be probabilistic)
\item CCA2: formal definition later
\end{enumerate}

Diffie-Hellman key exchange.
Alice, Bob. They pick a group $G$ (e.g. $(\mathbb{Z}/p \mathbb{Z})^*$ for some prime $p$) and $g \in G$.
Alice picks $a$. Bob picks $b$.

Alice send $g^a=\alpha$. Bob send $g^b=\beta$.
Alice computes $\beta^a$, Bob $\alpha^b$.
Which makes $g^{a b}$: OK.

Charlies sees $g^a$, $g^b$, $g$. Can he compute $g^{a b}$.

Presumably hard for $(\mathbb{Z}/p \mathbb{Z})^*$

(Unreasonable amount of computation : $2^{80}$ to $2^{128}$)

[To build a fully-secure cryptosystem, we would have to use a key longer than the message. And then, bitwise Xor]

\subsection{Formal definitions}

Encryption algorithm=(KeyGen,Enc,Dec) (KeyGen and Enc are probabilistic functions)

KeyGen: given a security parameter $\lambda$, generates $s_k$ and $p_k$.

Enc: Given $m$ a message in some plain text domain, and $p_k$ the public key, returns $c$.

Dec: Given $c$ and the secret key $s_k$, returns $m$ in the plain text domain.

\textit{Efficiency}: All three algorithms shold be probabilistic polunomial time.

\textit{Soundness}: With probabiblity "close to 1", Dec(Enc($m$,$p_k$),$s_k$)=$m$

\textit{Security assumptions}:
\textit{CPA (Chosen plaintext attack) security}:
\begin{itemize}
\item Attacker is given $p_k$
\item In a first step, he can generate as many ciphertexts as he wants
\item Second phase: He chooses $m_0$, $m_1$plaintexts and sends them
\item Third phase: He receives $C=Enc(p_k,m_{random(1)})$
\end{itemize}
The attacker must guess if $C$ is the encryption of $m_0$ or of $m_1$.

Scheme is said CPA-secure if $A$ cannot guess correctly with probability significantly better than $\frac{1}{2}$ (namely $\geq \frac{1}{2}+\frac{1}{poly(\lambda)}$)
\textit{CCA2 (Adaptative chosen ciphertext attack)}
\begin{itemize}
\item As before
\item $A$ can ask for decryption of any string of bits
\item As before
\item $A$ can ask for decryption of any string of bits $\neq c$
\end{itemize}
Same requirements.

Have access to a decryption oracle which will answer any question but the challenge.

Consequence: Ciphertext looks like random.

\subsection{GGH (Golreich Goldwasser, Haleri)}

Underlying problem: BDD/CVP.

Security parameter: $n$, dimension of a lattice

$s_k$: Good random basis of the lattice (almost orthogonal)($B_{s_k}$)

$p_k$: Bad basis of the same lattice ($B_{p_k}$)

\textit{Enc}:

Starting from a message $M \rightarrow B_{p_k} M+r$ r some small noise so that CVP($B_{p_k} M+r$)=$B_{p_k} M$

Attacker has to solve a CVP in a bad basis $\rightarrow$ hard.

\textit{Dec}: Solve CVP using Babai and the knowledge of $B_{s_k}$, or compute $B_{s_k} Round(B_{s_k}^{-1} C)$
$\rightarrow$ Recover $B_{p_k} M$
$\rightarrow$ Multiply by $B_{p_k}^{-1}$

Remark: You can exchange the roles of $m$ and $r$.

Practice: 
\begin{itemize}
\item Security hard to analyze precisely.
\item Unefficient. Keysize=$O(n^2)$. Underlying computations $O(n^2)$. Not very good.
\end{itemize}

\subsection{NTRU (Number Theory R Us)}

Same thing with structured matrices to lower keysize and computation time to (quasi-)linear in $n$.

Let $R$ be the ring $\mathbb{Z}[x]/(x^n-1)$.

$R_q := \mathbb{Z}/q \mathbb{Z}[x]/(x^n-1)$, for $q$ not too small, $q$ about $n^2$.

$p$ small prime-coprime to $q$ (say $p=3$).

$s_k$: $f, g \in R$ with coefficients in $\{0;1\}$
$f$ invertible in $\mathbb{Z}/p \mathbb{Z}[x]/(x^n-1)$
$g$ invertible in $\mathbb{Z}/q \mathbb{Z}[x]/(x^n-1)$

$p_k$: $h=\frac{g}{f}$, where $\frac{1}{g}$ is to be computed in $R_q$

Enc: $m \in R$ with very small coefficients, $\{0;1\}$. Sample $r \in R$ with very small coefficients. $C := p r h +M \in R_q$

Dec: $g^{-1} (C g mod p) mod p$

Soundness: $C g= p f r + g M (mod q)$. Which is small as sum and product of small things.
So $p f r+g M= Rem(C g,q))$

$\Rightarrow g M = Rem (C g, q) mod p$
$= Rem (Rem(C g,q) mod p)$
$M = Rem(g^{-1} Rem(Rem (C g, q),p),p)$

\begin{enumerate}
\item \textit{Why is it fast?}
Using FFT, multiplication in $R$ ($R_q$ if $q$ is well chosen) can be computed in $\tilde{O}(n)$
\item \textit{Why is it related to lattices?} $L$ the lattice spanned by the columns of $M$ (\textbf{Draw here the matrix - cf Hugo Feree}). Claim: $(g_0,...,g_{n-1},f_0,...,f_{n-1}) \in L$. Proof left as an exercise. Its norm is smaller then $\sqrt{2 n}$. Minkowski's bound for $L$ is about $\sqrt{2 n} (q^n)^{\frac{1}{2 n}}$ which is of the magnitude of $\sqrt{2 n} \sqrt{q}$. So it is about $n \sqrt{2 n}$. So the vector is way smaller than Minkowski's bound. So it is probably the shortest non-zero vector in $L$. Attacking the key is an SVP. Decoding can be seens as a BDD. About close to GGH, with a structured lattice, of keysize $\tilde{O}(n)$. Computational time is quasilinear: $\tilde{O}(n)$. $\Rightarrow$ Heuristic security
\end{enumerate}


\part{Algebraic number theory, Ideal lattices, Fully homomorphic encryption}

\section{Algebraic number theory and Ideal lattices}

\subsection{Cyclotomic fields and rings}

\begin{definition}
A primitive $n^{th}$ root of unity is an element of $\mathbb{C}^*$ of exact order $n$, namely $e^{\frac{2 i \pi k}{n}}$, where $(k,n)=1$.
\end{definition}

\begin{definition}
The $n^{th}$ cyclotomic polynomial $\Phi_n$  is $\Phi_n(X) = \Pi_{S \ prime \ n^{th} \ root \ of \ 1} (X-S)=\Pi_{(k,n)=1, 1 \leq k \leq n} (X - e^{\frac{2 i k \pi}{n}})$
$S_n=e^{\frac{2 i \pi}{n}}$
\end{definition}

\textbf{Check the ex with Hugo Feree}

rem: def $\Phi_n=phi(n)$
$\phi(n)=n \Pi_{p|n} (1-\frac{1}{p})$

($p$ is prime)

\begin{proposition}
$\Phi_n(X) \in \mathbb{Z}[X]$, monic.
\end{proposition}

\begin{proof}
\begin{lemm}
$X^n-1=\Pi_{d | n} \Phi_d(X)$
\end{lemm}
\begin{proof}
$X^n-1=\Pi_k (X-S_n^k)$
$k = k_1 k_2$, where $k_1=(k,n)$. $(k_2,n)=1$ $k_2=\frac{k}{k_1}$
Then $X^n-1= \Pi_{k_1 | n} \Pi_{1 \leq k_2 \leq \frac{n}{k_1}, (k_2,n)=1} (X-S_n^{k_1 k_2})$
$\Phi_{\frac{n}{k_1}}(X)=\Pi_{1}$
 \textbf{Proof here, check that with Hugo Feree}
\end{proof}
By induction, $\Phi_1(X)$
$\Phi_n(X)=\frac{X^n-1}{\Pi_{d | n d \neq n} \Phi_d(X)}$
By induction, $\Pi_{d | n, d \neq n} \Phi_d(X) \in \mathbb{Z}[X]$+monic.

The quotient of two monic polynomials with integer coefficients is monic, with integer coefficients.
\end{proof}

\begin{proposition}
$\Phi_n(X)$ is an irreducible polynomial over $\mathbb{Q}$
\end{proposition}
\begin{proof}
Admitted
\end{proof}

\begin{definition}
The $n^{th}$ cyclotomic field over $\mathbb{Q}$ is

$\mathbb{Q}(S_n)=\mathbb{Q}[x]/(\Phi_n(x)):=K_n$ (The first equality comes from the irreducibility of $\Phi_n$)
\end{definition}

Computationnally speaking, the representation in the most convenient.
\begin{itemize}
\item An element of $K_n$ is a polynomial with rationnal coeff. of degree $< n$
\item Addition is ordinary. Polynomial addition.
\item Mult is ordinary too, is polynomial mod $\Phi_n(X)$.
\item Inversion is doing the extended Euclidean algorithm on $p$ and $\Phi_n$ $P(X)U(X)+\Phi_n(X)V(X)=1$ (since $\Phi_n$ is irreducible) $\rightarrow P(X)U(X)=1 mod (\Phi_n(X))$
\end{itemize}

Rem: $P(X)$ corresponds to $P(S_n) \in \mathbb{Q}(S_n)$
Rem: $K_n$ has a $\mathbb{Q}$-vector space structure with dimension $\phi(n)$ 

\begin{definition}
The \textbf{ring} of integers $O_{K_n}$ of $K_n$ is $\mathbb{Z}[S_n]=\mathbb{Z}[X]/(\Phi_n(X))$
\end{definition}

Rem: $O_{K_n}$ is a free $\mathbb{Z}$-module with rank $\phi(n)$, with basis $1,S_n,...,S_n^{\phi(n)-1}$
Elments of $O_{K_n}$ are called \textbf{algebraic integers}.


\end{document}
