\documentclass[a4paper,10pt]{article}
\usepackage[utf8]{inputenc}
\usepackage{amsfonts}
\usepackage{amsmath}
\usepackage{amssymb}
\usepackage{proof}
\usepackage{amsthm}
\usepackage{cmll}
\usepackage{todonotes}
\usepackage{multicol}
\usepackage{stmaryrd}
\usepackage{hyperref}
\usepackage{fullpage}

\newcommand{\Nu}{\mbox{\Large $\nu $}} % dirty thing to have uppercase \nu ...

\hypersetup{
colorlinks=true, %colorise les liens
breaklinks=true, %permet le retour à la ligne dans les liens trop longs
urlcolor= blue, %couleur des hyperliens
linkcolor= red, %couleur des liens internes
citecolor=blue, %couleur des références
} 

\author{Olivier Laurent \and Daniel Hirschkoff \\ \and (transcribed by the students)}

\title{Linear logic and concurrency.} 

\newcommand{\ie}{\textit{i.e.}\ }
\newcommand{\cf}{\textit{c.f.}\ }
\newcommand{\eg}{\textit{e.g.}\ }
\newcommand{\parc}{\ |\ }
\newcommand{\R}{{\mathcal R}}
\newcommand{\interp}[1]{\llbracket #1\rrbracket}

\newtheorem{definition}{Definition}
\newtheorem{prop}{Property}
\newtheorem{theo}{Theorem}
\newtheorem{lemma}{Lemma}
\newtheorem{ex}{Example}
\newtheorem{exo}{Exercise}
\newtheorem{rmk}{Remark}

\begin{document}

\maketitle
\part{Linear Logic}
\section{Multiplicative Linear Logic (MLL)}

$A ::= X | X ^ \perp | A \otimes A | A \parr A$

\begin{definition}
	\begin{itemize}
		\item $(X)^\perp := X ^\perp$
		\item $(X^\perp)^\perp := X$
		\item $(A \otimes B) ^\perp := A^\perp \parr B^\perp$ 
		\item $(A \parr B)^\perp := A ^\perp \otimes B ^\perp$
	\end{itemize}	
\end{definition}

\begin{prop}
	$\forall A, (A ^\perp)^\perp = A$
\end{prop}

\textbf{Notation:} $A\multimap B :=  A ^\perp \parr B$

\textbf{Rules:}
\begin{center}
$	\infer[ax]{\vdash A^\perp, A}{} \quad
	\infer[cut]{\vdash \Gamma, \Delta}{\vdash A, \Gamma & \vdash A^\perp, \Delta} \quad
	\infer[\otimes]{\vdash A \otimes B, \Gamma, \Delta}{\vdash A, \Gamma & \vdash B, \Delta} \quad
	\infer[\parr]{\vdash A \parr B, \Gamma}{\vdash A, B, \Gamma} \quad
	\infer[ex]{\vdash \Gamma \sigma}{\vdash \Gamma} 
$
\end{center}
	
\begin{ex} \label{ex:comm_par} $\vdash (A\otimes B)^\perp, (B\otimes A)$

$$\infer[\parr]{\vdash A ^\perp \parr B^\perp, B \otimes A}
	{\infer[ex]{\vdash A^\perp,B^\perp,B\otimes A}
		{\infer[\otimes]{\vdash B\otimes A, B^\perp, A^\perp}
			{\infer[ax]{\vdash B, B^\perp}{} & \infer[ax]{A, A^ \perp}{}
			}}}$$
\end{ex}

\textbf{Difference with classical logic:}
There is no duplication. Formulas are used once.

\begin{ex}
	$A \vdash A \land A$, is provable in classical logic, but $\vdash A\otimes A, A ^\perp$ is not provable in MLL.
\end{ex}


\begin{exo}
	Which of these sequents are provable?
	
	\begin{multicols}{2}
	\begin{itemize}
		\item $A \otimes A\vdash A$
		\item $A\vdash A\parr A$
		\item $A \parr B\vdash A$
		\item $A\otimes B\vdash A$
		\item $A \parr B\vdash B \parr A$
		\item $A \otimes (B\parr C)\vdash (A\otimes B)\parr C$
		\item $A \parr (B\otimes C)\vdash (A\parr B)\otimes C$
		\item $(A\otimes B)\parr C \vdash A \otimes (B \parr C)$
		\item $(A\otimes B)\otimes C \vdash A \otimes (B \otimes C)$
	\end{itemize}	
	\end{multicols}
\end{exo}

\section{Proof-nets}

\subsection{Proof-structures:}

\begin{figure}[h]
\begin{center}
	\includegraphics{figures/parr.pdf} \ 
	\includegraphics{figures/tens.pdf} \ 
	\includegraphics{figures/ax.pdf} \ 
	\includegraphics{figures/cut.pdf} \ 
\end{center}
\end{figure}

\begin{ex}
	\begin{figure}[h]
	\begin{center}
		\includegraphics{figures/pn_ex1.pdf}
		\caption{Example of translation from proof (of example~\ref{ex:comm_par}) to proof-structure.}
	\end{center}
	\end{figure}
\end{ex}

\begin{definition}[Proof-net]
	A proof net is a proof structure obtained from a proof.
\end{definition}
\begin{ex}
The following proof structure is not a proof-net:
		$$\includegraphics{figures/pn_ax_tens.pdf}$$
\end{ex}

The transformation from proof to proof-structure is then not surjective. One reason is that in proof structures, there is no difference between $\otimes$ and $\parr$.

\subsection{Correctness criterion}

\begin{definition}[Correction graph]
	A correction graph of a proof structure is obtained by removing one of the two upper edges of each $\parr$-node.
\end{definition}

\begin{prop}[Correctnesss criterion (Danos-Regnier)]
	A proof structure is a proof-net if and only if all its correction graphs are connected and acyclic.
\end{prop}
\begin{proof}
	\textbf{Correction:} $\Pi \vdash \Gamma \Rightarrow {\Pi^*\over \Gamma} \in PN$
	
	This is the easy part, by induction on the derivation.
	
	\textbf{Sequentialization:} ${\mathcal R} \in PN \Rightarrow \exists \Pi \vdash \Gamma, \Pi^* = \mathcal R$
	
	By induction on the size of $\mathcal R$.
	\begin{enumerate}
			\item If there is a $\parr$ node above a conclusion:
			$$\includegraphics{figures/sequentialization1.pdf}$$
			
			Then, since $\mathcal R_0$ verifies the acyclic and connected condition, there is a proof $\Pi_0$ of $\vdash A, B, \Gamma$. Then we have the proof 
			$$\infer[\parr]{\vdash A\parr B, \Gamma}{\infer{\vdash A, B, \Gamma}{\Pi_0}}$$
			
			\item If there are cuts in $\mathcal R$, replace them with $\otimes$ nodes: 
			$$\includegraphics{figures/sequentialization2.pdf}$$
			This does not change the connected and acyclic condition.
			
			\item If $\mathcal R_0$ does not contain any cut, has no $\parr$ node above its conditions, and has at least one $\otimes$ node above a condition, then the splitting lemma (admitted) shows that we can separate $\mathcal R_0$ this way: 
			
			$$\includegraphics{figures/sequentialization3.pdf}$$
			Then, $\mathcal R_1$ and $\mathcal R_2$ verify the conditions, and by induction hypothesis, are associated to some proofs $\Pi_1$ and $\Pi_2$, which give the following proof: $$\infer[\otimes]{\vdash \Gamma A\otimes B, \Delta}{\infer{\vdash \Gamma, A}{\Pi_1} & \infer{\vdash \Delta, B}{\Pi_2}}$$
			
			\item Else, there are only axiom nodes, but the connectedness condition implies that there is only one, and the corresponding proof is trivial.
	\end{enumerate}
	
		
	
\end{proof}

\subsection{Cut elimination}
	\begin{figure}[h]
	\begin{center}
		\includegraphics{figures/ce_ax.pdf}
		\caption{$ax$ reduction rule.}
		\includegraphics{figures/ce_par_tens.pdf}
		\caption{$\parr - \otimes$ reduction rule.}
		
	\end{center}
	\end{figure}	
	
	Note that the reduced proof-structure is still a proof-net.
	
	\begin{prop}
	Strong normalization, confluence.
	\end{prop}	
	
	\begin{prop}[Sub-formula property]
	$$\includegraphics{figures/subformula.pdf}$$
	\end{prop}	
	
	\begin{proof}
		Use cut elimination.
	\end{proof}
	
	\begin{rmk}
	This implies that $\emptyset$ is not provable.
	\end{rmk}
	
	
	\subsection{Exponential connectives}
	
	MLL is not sufficient, computationally speaking, for intuitionistic logic.
	
	\begin{ex}
		$A \rightarrow B \rightarrow A$ is provable in intuitionistic logic, but $A \multimap B \multimap A$ is not provable in MLL.
	\end{ex}	
	
	\begin{definition}[MELL rules:] MLL rules +
	
	$\infer[\wn_c]{\vdash \Gamma, \wn A}{\vdash \Gamma, \wn A, \wn A}$ (contraction) 
	$\infer[\wn_w]{\vdash \Gamma, \wn A}{\vdash \Gamma}$ (weakening)
	
	$\infer[\wn_d]{\vdash \Gamma, \wn A}{\vdash \Gamma, A}$ (dereliction)
	$\infer[\oc]{\vdash \wn\Gamma, \oc A}{\vdash \wn\Gamma, A}$(promotion)
	\end{definition}
	
	\begin{exo}
		$$\begin{array}{l l l l l}
			LK & \overset{\lnot\lnot}{\longrightarrow} &  LJ & \longrightarrow & MELL \\
			& & \Gamma \vdash A & & \oc \Gamma ^* \vdash A^* \equiv\ \vdash \wn(\Gamma ^ *)^\perp, A^*
		\end{array}$$
		where $(A \rightarrow B)* = \oc  A^* \multimap B$		
	\end{exo}	

	\begin{figure}[h]
	\begin{center}
		\includegraphics{figures/mell_c.pdf} \quad
		\includegraphics{figures/mell_w.pdf} \quad
		\includegraphics{figures/mell_d.pdf}	\quad
		\includegraphics{figures/mell_p.pdf}	\quad
		\caption{The nodes corresponding to the contraction, weakening, dereliction and promotion rules.}
	\end{center}	
	\end{figure}	
	
	Note that the promotion node is not sufficient in itself to express the promotion rule (\ie the context must be of the form $\wn\Gamma$).
	The solution is to add boxes (one box per $\oc $-node):
	\begin{figure}[h]
	\begin{center}
		\includegraphics{figures/box.pdf}
		\caption{Re presentation of a $\oc $-box.}
	\end{center}	
	\end{figure}
	
	Connectedness of correction graphs is not true anymore. In fact we have:
	
	\begin{prop}
		For any correction graph of a proof-net, we have: \\ $\sharp$ connected components $=1+ \sharp \wn-$nodes.
	\end{prop}	
	
	This condition is not sufficient, as shown by the following example:
	
	\begin{ex}
		The correctness graphs of the following example verify the previous property ($2$ connected components), but is not a proof-net.
	$$\includegraphics[height=0.3\textheight]{figures/connected.pdf}$$
	\end{ex}
	
	\begin{rmk}
		There is no non-trivial overlapping of boxes.
	\end{rmk}	
	
	This allows us to define the depth of a $\oc $-box by the number of box containing it.
	
	\begin{prop}[Acyclicity criterion]
		In a proof-net, correction graphs are acyclic (where correction graphs are now obtained by removing one of the upper edges of each $\parr$-node and each $\wn_c$-node.
	\end{prop}	
	
	These properties can be summed up in the following theorem:
	
	\begin{theo}
		$\Pi \vdash_{MELL} \Gamma \Rightarrow$ $\Pi^*$ (and recursively all its boxes) has acyclic correction graphs, and $\sharp cc = \sharp \wn_w +1$.
	\end{theo}	
	
	As stated before, the converse is false.
	
	\begin{prop}[Sequentialization]
		If $\mathcal R$ is $\wn_w$-free with acyclic connected correction graphs, then $\exists \Pi, \Pi* = \mathcal R$.
	\end{prop}	
	
	\begin{exo}
		\begin{multicols}{2}
		\begin{itemize}
			\item $\oc (A\otimes B)\overset{\dashv}{\vdash} \oc A \otimes \oc B$
			\item $\oc \oc A \overset{\dashv}{\vdash} \oc A$
			\item $\wn\wn A \overset{\dashv}{\vdash} \wn A$
			\item $\oc A \otimes \oc A\overset{\dashv}{\vdash} \oc A$
			\item $\oc A \otimes \oc B\overset{\dashv}{\vdash} \oc A$
			\item $\wn A \otimes \wn A\overset{\dashv}{\vdash}\wn A$
			\item $\wn A \parr \wn A\overset{\dashv}{\vdash}\wn A$
			\item $\oc A\overset{\dashv}{\vdash}\wn A$
			\item $\oc A\overset{\dashv}{\vdash}A$
		\end{itemize}	
		\end{multicols}
	\end{exo}	
	
	\begin{definition}[Modality]
		Only arity-1 connectives: $\underset{\mu}{\underbrace{\oc \dots \wn \dots \oc  \dots}}A$
		
		$\mu \sim \nu \Leftrightarrow \forall A, \mu A \overset{\dashv}{\vdash} \nu A$
		
		$\mu \le \nu \Leftrightarrow \forall A, \mu A \vdash \nu A$
	\end{definition}	
	
	\begin{exo}
		Determine $\le$ and $\sim$.
	\end{exo}
	
	\subsection{Cut elimination}
	\begin{rmk}
		A cut from a $\oc $-node to a $\wn $-node of the corresponding box is impossible (since the graph of the corresponding box is connected, it would introduce a cycle).
		\end{rmk}		
	The only additional cuts are of the form, and can be reduced as described by the following figures:% figures~\ref{fig:ce_d}, \ref{fig:ce_w}, \ref{fig:ce_c} and \ref{fig:ce_box}
	
		$$\includegraphics[scale=0.55]{figures/ce_d.pdf}$$
		$$\includegraphics[scale=0.55]{figures/ce_c.pdf}$$
		$$\includegraphics[scale=0.55]{figures/ce_w.pdf}$$
		$$\includegraphics[scale=0.55]{figures/ce_box.pdf}$$
	
	\begin{exo}
		Check that the correction criterion is preserved through cut reduction.
	\end{exo}	
	
	\begin{prop}[Local confluence]
	$$\includegraphics[scale=0.5]{figures/confluence.pdf}$$
	\end{prop}	
		
	\begin{rmk}
		In fact we have confluence (but this is a hard result).
	\end{rmk}
		

	\begin{prop}[Termination]
	\begin{description}
		\item[Weak normalization:] $\forall {\mathcal R}, \exists {\mathcal R_0}$ cut-free  such that ${\mathcal R} \rightarrow^* {\mathcal R_0}$.
		\item[Strong normalization] $\forall {\mathcal R}$, each reduction chain begining by ${\mathcal R}$ is finite.
	\end{description}		
	\end{prop}
	\begin{proof}[Proof of weak normalization:]
	 There are two kinds of cuts: exponential and multiplicative cuts: 
	 $$\includegraphics{figures/exp_cut.pdf}\qquad \includegraphics{figures/mult_cut.pdf}$$
	 We give to an exponential cuts the measure $(|A|, |\wn -tree|)$ where $\wn -tree$ is the tree above $\wn A$ of the form: $$\includegraphics[width=0.3\textwidth]{figures/exp_tree.pdf}$$
	 
	 We give to multiplicative cuts the measure $(|A\parr B|, 0)$.
	 
	 We now want to show that there is a reduction such that the multi-order (based on the lexicographic order) of the multi-set of these measures decreases.
	 
	 \begin{prop}%todo ? define paths ?
		If there are no multiplicative cuts, then the paths of the proof-net are acyclic.
	 \end{prop}
	 \begin{proof}
		Assume that there is a cycle (take a smallest one). Since the depth decreases, it is constant. This cycle can not go down in a branch of $\parr$ node and go up in the other branch of the same node (or is not minimal). Then, the path is in a correction graph, which is acyclic.
	 \end{proof}				
	 This result allows us to define an order on cuts: a cut is smaller than another one if there is a path from the second one to the first.
	 We consider a maximal cut of maximal depth. Reducing this cut makes the (multi-set) measure decreases.
	 %TODO: to finish
	 \end{proof}
	 
	 \part{Process calculi}

\section{A basic process calculus}

channels, names : $a, b, c\dots$
Each name $a$ has a co-name $\bar a$

Processes: $P::= (P\parc P) \parc \eta \parc 0$
where $\eta = a$ or $\bar a$, $O$ is the inactive process.

\subsection{Operational semantics: reduction.}

\begin{definition}[Structural congruence: $\equiv$]
	It is an equivalence relation such that: 
	$$\infer{P \parc T \equiv Q \parc T}{P \equiv Q} \quad  \infer{P\parc 0  \equiv 0}{}\quad  \infer{P\parc Q \equiv Q \parc P}{}\quad \infer{P \parc (Q \parc R) \equiv (P \parc Q) \parc R}{}\footnote{The last two are soup laws.}$$
\end{definition}	
	\begin{ex}
		$a \parc (b \parc \bar a) \equiv b \parc a \parc \bar a$
	\end{ex}
	
	\textbf{Reduction:} $infer{a \parc \bar a \rightarrow 0}{}$
	Interaction is basically synchronization.
	$$\infer{P \equiv Q}{P \equiv P_1 & P_1 \rightarrow Q_1 & Q_1 \equiv Q}\quad \infer{P \parc Q \rightarrow P'\parc Q}{P\rightarrow P'}$$
	
	\begin{ex}
		$\infer{a \parc b \parc \bar a \rightarrow b}
		{a \parc b \parc \bar a \equiv a \parc \bar a \parc b & \infer{a \parc \bar a \parc b \rightarrow 0 \parc b}{\infer{a \parc \bar a \rightarrow 0}{}}
		& b \equiv b}$
	\end{ex}
	
	\begin{rmk}
		In $a\parc \bar a \parc a$, we don't know which $a$ was reduced.
	\end{rmk}

\subsection{Operational semantics: labeled transition system.}
	Judgments: $P \xrightarrow \mu  P'$. Actions: $\mu = a\parc \bar a \parc \tau$ ($\tau$ is a special constant).
	
	$$
		\infer{a \xrightarrow a 0}{}\quad
		\infer[C_1]{P \parc Q \xrightarrow \tau P' \parc Q'}{P \xrightarrow a P' & Q \xrightarrow {\bar a} Q'}\quad
		\infer[P_1]{P\parc Q \xrightarrow \mu P'\parc Q}{P\xrightarrow \mu P'}\quad
	$$
	$$
	\infer{\bar a \xrightarrow {\bar a} 0}{}\quad
	\infer[C_2]{P \parc Q \xrightarrow \tau P' \parc Q'}{P \xrightarrow {\bar a} P' & Q \xrightarrow a Q'}	\quad{}
	\infer[P_2]{P\parc Q \xrightarrow \mu P\parc Q'}{Q\xrightarrow \mu Q'}
	$$
	There are no soup laws anymore.
	
	\begin{lemma}[Harmony lemma]\label{lem:harmony}
		$P\xrightarrow \tau P'$ if and only if there exists $P_1$ such that $P\rightarrow P_1$ and $P_1 \equiv P'$.
	\end{lemma}
	
	
\subsection{Behavioural equivalences}

\begin{definition}[Bisimilarity $\sim$]
	$\sim$ is the greatest symmetric relation such that whenever $P \sim Q$, if $P \xrightarrow \mu P'$, then there exists $Q'$ such that $Q\xrightarrow \mu Q'$ and $P' \sim Q'$: $$\includegraphics{figures/bisimilarity.pdf}$$
\end{definition}

	\textbf{Proof technique: bisimulation.} Find a relation $\mathcal R$ such that $$\includegraphics{figures/bisimulation.pdf}$$ Then $\mathcal R \subseteq \sim$ .

	\begin{prop}
		$\sim$ is an equivalence relation.
		Symmetry is ok.
		
		Relfexivity: take ${\mathcal R} = \{(P, P)\}$.
		
		Transitivity: $$\includegraphics{figures/bis_trans.pdf}$$
	\end{prop}
	
	\begin{lemma}[Congruence]
		If $P\sim Q$, then $\forall T, P \parc T \sim Q\parc T$.
	\end{lemma}
	\begin{proof}% TODO or not to do ?
		${\mathcal R} = \{(P \parc T, Q \parc T) \parc P\sim Q\}$. Prove that $\mathcal R$ is a bisimulation, by a case analysis ($C_1, C_2, P_1, P_2$).
	\end{proof}
	
	\begin{definition}
		Let $\mathcal R$ be a relation between processes. $\mathcal R$ is:
		\begin{enumerate}
			\item Reduction closed if $ \left . \begin{matrix} P \R Q \\ P \rightarrow P' \end{matrix}\right \} \Rightarrow \exists Q'	
			\left \{ \begin{matrix} P' \R Q' \\ Q \rightarrow Q'\end{matrix} \right .$.
				\label{one}
			\item Context closed if  $P\R Q \Rightarrow  \forall T, (P \parc T) \R (Q \parc T)$.
			\label{two}
			\item Observable preserving if "$a$ appears in $P$" and $P\R Q$ implies that "$a$ appears in $Q$".
			\label{three}
		\end{enumerate}
	\end{definition}
	
	\begin{definition}[Barbed equivalence $\simeq$]
		$\simeq$ is the greatest symmetric relation verifying \ref{one}, \ref{two} and \ref{three}.
	\end{definition}
	
	%\begin{ex}/ TODO true ???
		%$0  \not \simeq \bar a$ (but $0 \sim \bar a$)
	%\end{ex}
	
	\begin{lemma}\label{lem:equiv_sim}
		$\equiv \subseteq \sim$
	\end{lemma}
	\begin{proof}
		Easy.
	\end{proof}
	
	\begin{theo}
	$P \sim Q \Leftrightarrow P \simeq Q$
	\end{theo}
	\begin{proof}
		$\sim \subseteq \simeq$:
		
		\begin{enumerate}
				\item $\sim$ is reduction closed:
				Suppose that $P \sim Q$ and $P\rightarrow P'$. Then, by the harmony lemma, $P \xrightarrow \tau P'$, and by definition of $\sim$, there exists $Q'$ such that $Q \xrightarrow Q'$ and $P' \sim Q'$. By the other implication of the harmony lemma, there exists $Q_1$ such that $Q_1 \equiv Q'$ and $Q\rightarrow Q_1$. Then, lemma~\ref{lem:equiv_sim} allows us to conclude.
				
				\item $\sim$ is closed by context: it has already been proved.
				
				\item If $a$ appears in $P$ and $P\sim Q$, then $a$ appears in $Q$.
				\begin{lemma}[admitted]
					If $a$ appears in $P$, then $P\xrightarrow a P'$ for some $P'$.
				\end{lemma}
				By this lemma, there exists $Q'$ such that $P' \sim Q'$ and $Q \xrightarrow a Q'$, which means that $a$ appears in $Q$.
		\end{enumerate}
		
		$\simeq \subseteq \sim$: Suppose that $P \simeq Q$ and $P \xrightarrow a P'$.
		Then we have: $$\includegraphics{figures/barbed_is_bisim.pdf}$$ where $\tau$ "corresponds to an $a$" and $Q \xrightarrow a Q'$.
		For this we need the following (admitted) lemma:
		\begin{lemma}
			If $P \simeq Q$, then $|P|_{a} = |Q|_{a}$.
		\end{lemma}		
	\end{proof}
	
\section{Toward CCS: prefixing}

$P ::= (P_1 \parc P_{2}) \parc 0 \parc \eta . P$ where $\eta = a \parc \bar a$.

Interaction $=$ synchronization and the triggering of continuations: $\infer{a.P \parc \bar a .Q \rightarrow P \parc Q}{}$ and the other rules. 

\textbf{Notation:} $\eta .0$ is noted $\eta$, and $a$ is $a.0$.


\begin{ex}
	$\bar a \parc a.b \parc \bar b \parc a {\nearrow \atop \searrow} \begin{matrix} b\parc \bar b \parc a \rightarrow a \\ a.b \parc \bar b \not\rightarrow\end{matrix}$
\end{ex}

\subsection{Labeled transition system}

$\infer{a.P \xrightarrow a P}{}$ $\infer{\bar a.P \xrightarrow {\bar a} P}{}$

\begin{ex}
	$\begin{matrix}a.b &\not \sim & a \parc b \\ \not \downarrow_{b} & & \downarrow_{b} \end{matrix}$, but $a.a \sim a\parc a$. 
	Indeed, $\mathcal R \{(a.a, a \parc a), (a, a \parc 0), (a, 0\parc a), (0, 0\parc 0)\}$ is a bisimulation.
\end{ex}

\begin{rmk}
	$a.P\parc a.P \sim a.(P \parc a.P)$
\end{rmk}

\subsection{Behavioural equivalences}

We explicit the third condition of the barbed equivalence: $a$ appears in $P$ if $P \equiv a.P_{1} \parc P_{2}$.

\begin{theo}
	$\sim = \simeq$
\end{theo}
\begin{proof}
	$\begin{matrix} P & \simeq & Q \\ \downarrow_{a} & & \\ P' & & \end{matrix}$
	Let $c$ be a fresh name. 
	By definition of $\simeq$, $P\parc \bar a. \parc c.0 \parc c.1000 \simeq Q\parc \bar a. \parc c.0 \parc c.1000$, where $1000$ represents a process which has a big sequence of reduction (big compared to the size of the reduction sequences beginning from $P$ or $Q$).%
	The left one reduces to $P' \parc \bar a. \parc c.0 \parc c.1000$. This enforces $Q$ to interact with $\bar a$. %TODO: finish
\end{proof}

\section{Finite public $\pi$-calculus}

This calculus adds name passing and substitution.

$P ::= (P_{1} \parc P_{2}) \parc 0\parc \eta.P$, with $\eta = \bar a<b> \parc a(b)$

$a(b)$: input on $a$ of channel $b$

$\bar a <b>$: output of $b$ on $a$.

$\infer{a(x).P \parc \bar <b> .Q \rightarrow P[b/x]\parc Q}{}$


\begin{ex}
	\begin{itemize}
			\item forwarder: $a(x).\bar b <x>.0$ (receives $x$ on $a$ and forwards it on $b$)
			\item pointers: $\bar a <c>. \bar c <t> .0 \parc a(x).x(y).P \rightarrow \bar c<t>.0 \parc c.(y).P \rightarrow 0.\parc P[c/x][t/y]$
			\item if then else: $a(x).(\bar x <t> | b(y).P_{1} | c(z).Q_{2})$
	\end{itemize}
\end{ex}

%LTS: later (?)

\textbf{Notation} $\bar a<b>.P$ can also be noted $\bar a b.P$.

\subsection{Diadic $\pi$-calculus}
$\eta = a(x, y) \parc \bar a <x, y>$

There is an encoding $\interp{}$ of $\Pi_{2}$ into $\Pi_{1}$:
% give the first ideas ?
$\interp{\bar a <b, c>.P} = \bar a n.\bar n b.\bar n c.\interp{P}$ with $n$ fresh.
$\interp{a(x,y).Q} = a(p).p(x).p(y).\interp{Q}$ with $p$ fresh in $Q$.

\section{Finite $\pi$-calculus}

$P=0 \parc (P_1 \parc P_2) \parc \eta P \parc (\Nu c) P$
with $\eta = a(x) \parc \bar a<b>$

$\Nu c$ is a restriction (such as the input): in $\Nu c P$, $c$ is private to $P$ (\ie only known by $P$).

This restriction is a binder: $(\Nu c) P \equiv_\alpha (\Nu c') P[c'/c]$ if $c'\not \in fn(P)$. \footnote{$fn(P)$ is the set of names which are free in $P$.}

%only the first one (?)
\begin{prop}
\begin{itemize}
	\item scope extrusion: $a(x).P \parc (\Nu c) \bar a<c>. Q \rightarrow \Nu c (P[c/x]\parc Q)$
	\item $P \parc (\Nu c) Q \equiv (\Nu c) (P \parc Q)$ if $c\not \in fn(P)$
	\item $(\Nu c)(\Nu d) P \equiv (\Nu d)(\Nu c) P$ in general, we write $(\Nu \tilde c)P$, with $\tilde c$ a tuple of names.
\end{itemize}
\end{prop}

\subsection{Reduction}
The same plus: $\infer{(\Nu c) P \rightarrow (\Nu c) P'}{P \rightarrow P'}$

\begin{proof}{of scope extrusion}
$$\infer[\equiv]{a(x).P\parc (\Nu c) \bar a<c>.Q \rightarrow (\Nu c)(P[c/X] \parc Q)}
	{
		\infer[\Nu]{(\Nu c)(a(x).P\parc \bar a<c>.Q)\rightarrow (\Nu c)(P[c/x]\parc Q)}	
		{
			\infer{a(x).P\parc \bar a<c>.Q\rightarrow P[c/x]\parc Q}{}
		}
	}
$$

\end{proof}	

\subsection{Diadic communication}

$\interp{\bar a<b, c>.P} = (\Nu n)(\bar a<n>.\bar n<b>.\bar n<c>.\interp{P})$
$\interp{a(b, c).Q} = a(t).t(x).t(y).\interp{Q}$

%\begin{ex} 
%hello, ack, key ... TODO ?
%\end{ex}

Here we have a lot of interactions: synchronisation, triggering of continuation, name passing and scope extrusion\footnote{The last two represent name mobility.}.

\begin{ex}
	If $c \not \in fn(P)$, $(\Nu c) P \equiv P$:
$(\Nu c) P \equiv (\Nu c)(P\parc 0) \equiv P\parc (\Nu c) 0 \equiv P\parc 0\equiv P$\footnote{we need to add the rule $(\Nu c) 0 \equiv 0$}
\end{ex}

\subsection{Behavioural equivalences}

\textbf{Barbed equivalence:}
\begin{enumerate}
	\item Nothing to change.
	\item contexts: $C[] ::= ([]\parc T)\parc (T\parc []) \parc (\Nu c)[]\parc a(x).[]$
	\item observability: $P\downarrow_a = (\Nu \tilde c)(a(x).P_1\parc P_2)$
for some $\tilde c$, $P_1$ and $P_2$ such that $a\not\in \tilde c$\footnote{and the similar definition with $\bar a$.}
\end{enumerate}

\begin{prop}[behavioural law]
	$$(\Nu t )(\bar a <t>.\bar t<c>.Q)\simeq (\Nu t)(\bar a<t> \parc \bar t<c>.Q)$$
\end{prop}

\section{Asynchronous $\pi$-calculus ($A\Pi$)}

$P ::= 0 \parc (P_1\parc P_2)\parc a(x).P \parc \bar a b$: there is no continuation after an output. $\bar a b$ can be viewed as a message.

\begin{rmk}
	$A\Pi \subseteq \Pi$ ($\bar a b$ is equivalent to $\bar a b.0$)
\end{rmk}

\begin{exo}{Program $\Pi$ into $A\Pi_2$}
	$\interp{\bar a b.P} =  \Nu r(\bar a<b, r>.r(z).\interp{P})$
with $z, r$ fresh for $P$
	$\interp{a(x).Q} = a(x, z).(\bar z<t>.\interp{Q})$
with $z$ fresh for $Q$
\end{exo}

\section{$\pi$-calculus}

$P ::= \dots  \parc !a(x).P$

$!$ is the replication operator.

$\interp{!a(x).P\parc \bar a<b>.Q \rightarrow !a(x).P \parc P[b/x]\parc Q}{}$

\begin{ex}
		$!a(x).\bar a<x> \parc \bar a<b>$ loops.\footnote{You can also do a nuclear bomb: $!a(x).(\bar a <x> \parc \bar a<x>)\parc \bar a<b>$}
\end{ex}

\subsection{Encoding $\lambda$ in $\Pi$}
Intuition: let $f = x\mapsto M$ is $!f(x).\interp{M}$

Let's take a reduction strategy: we chose call by value.

Terms: $M=x \parc \lambda x. M \parc M_1\ M_2$

Values: $V::= x \parc \lambda x.M$

\textbf{Reduction:}

$
	\infer{(\lambda x.M)\ V \rightarrow M[V/x]}{}\quad
	\infer{M\ N \rightarrow M'\ N}{M\rightarrow M'}\quad
	\infer{V\ N \rightarrow V\ N'}{N \rightarrow N'}
$
If $p$ is a $\Pi$-calculus name, $\interp{M}_p$ is defined by:

\begin{itemize}
	\item $\interp{x}_p =  \bar p x$ (injection of $\lambda$-calculus variables into $\Pi$-calculus names)
	\item $\interp{\lambda x.M}_p = (\Nu y)(\bar p y \parc !y(x,q).\interp{M}_q)$
	\item $\interp{M\ N}_p=(\Nu q)(\interp{M}_q \parc q(f).((\Nu r)(\interp{N}_r \parc r(x).f<x,p>)))$
\end{itemize}
That is first compute $M$, then $N$, then perform the $\beta$-reduction.

\begin{ex}
	$\interp{\lambda x.x}_p = (\Nu y)(\bar p y \parc y(x,q).\bar q x)$
\end{ex}

\begin{rmk}
	Note that there are two kind of names: $x, y, f$ are values, and $p, q, r$ are pointers to values.

In fact we gave an encoding into $A\Pi_2$
\end{rmk}

%todo or not todo? A\Pi_2 into A\Pi

How to define a LTS such that $\simeq$ and $\sim$ are equivalent?

\subsection{Weak behavioural equivalence}

We denote by $\Rightarrow$ the reflexive and transitive closure of $\rightarrow$.

The weak barbed congruence ($\cong$) is the greatest symmetric relation such that:

\begin{enumerate}
	\item %HERE ...
	\item
	\item 
\end{enumerate}

\end{document}

	
