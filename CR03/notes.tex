\documentclass[a4paper,10pt]{article}
\usepackage[utf8]{inputenc}
\usepackage{amsfonts}
\usepackage{amsmath}
\usepackage{amssymb}
\usepackage{proof}
\usepackage{amsthm}
\usepackage{cmll}
\usepackage{todonotes}
\usepackage{multicol}



\author{Olivier Laurent \and Daniel Hirschkoff \\ \and (transcribed by the students)}

\title{Linear logic and concurrency.} 

\newcommand{\ie}{\textit{i.e.}\ }
\newcommand{\cf}{\textit{c.f.}\ }
\newcommand{\eg}{\textit{e.g.}\ }

\newtheorem{definition}{Definition}
\newtheorem{prop}{Property}
\newtheorem{theo}{Theorem}
\newtheorem{ex}{Example}
\newtheorem{exo}{Exercise}
\newtheorem{rmk}{Remark}

\begin{document}

\maketitle
\part{Linear Logic}
\section{Multiplicative Linear Logic (MLL)}

$A ::= X | X ^ \perp | A \otimes A | A \parr A$

\begin{definition}
	\begin{itemize}
		\item $(X)^\perp := X \perp$
		\item $(X^\perp)^\perp := X$
		\item $(A \otimes B) ^\perp := A^\perp \parr B^\perp$ 
		\item $(A \parr B)^\perp := A ^\perp \parr B ^\perp$
	\end{itemize}	
\end{definition}

\begin{prop}
	$\forall A, (A ^\perp)^\perp = A$
\end{prop}

\textbf{Notation:} $A\multimap B :=  A ^\perp \parr B$

\textbf{Rules:}
\begin{center}
$	\infer[ax]{\vdash A^\perp A}{} \quad
	\infer[cut]{\vdash \Gamma, \Delta}{\vdash A, \Gamma & \vdash A^\perp, \Delta} \quad
	\infer[\otimes]{\vdash A \otimes B, \Gamma, \Delta}{\vdash A, \Gamma & \vdash B, \Delta} \quad
	\infer[\parr]{\vdash A \parr B, \Gamma}{\vdash A, B, \Gamma} \quad
	\infer[ex]{\vdash \Gamma \sigma}{\vdash \Gamma} 
$
\end{center}
	
\begin{ex} \label{ex:comm_par} $\vdash (A\otimes B)^\perp, (B\otimes A)$
\todo[inline]{TODO}
\end{ex}

\textbf{Difference with linear logic:}
There is no duplication. Formulas are used once.

\begin{ex}
	$A \vdash A \land A$, is provable in classical logic,but $\vdash A\otimes A, A ^\perp$ is not provable in MLL.
\end{ex}


\begin{exo}
	Which of these sequents are provable?
	
	\begin{multicols}{2}
	\begin{itemize}
		\item $A \otimes A\vdash A$
		\item $A\vdash A\parr A$
		\item $A \parr B\vdash A$
		\item $A\otimes B\vdash A$
		\item $A \parr B\vdash B \parr A$
		\item $A \otimes (B\parr C)\vdash (A\otimes B)\parr C$
		\item $A \parr (B\otimes C)\vdash (A\parr B)\otimes C$
		\item $(A\otimes B)\parr C \vdash A \otimes (B \parr C)$
		\item $(A\otimes B)\otimes C \vdash A \otimes (B \otimes C)$
	\end{itemize}	
	\end{multicols}
\end{exo}

\section{Proof-nets}

\subsection{Proof-structures:}

\begin{figure}[h]
\begin{center}
	\includegraphics{figures/parr.pdf} \ 
	\includegraphics{figures/tens.pdf} \ 
	\includegraphics{figures/ax.pdf} \ 
	\includegraphics{figures/cut.pdf} \ 
\end{center}
\end{figure}

\begin{ex}
	\begin{figure}[h]
	\begin{center}
		\includegraphics{figures/pn_ex1.pdf}
		\caption{Example of translation from proof (of example~\ref{ex:comm_par}) to proof-structure.}
	\end{center}
	\end{figure}
\end{ex}

\begin{definition}{Proof-net}
	A proof net is a proof structure obtained from a proof.
\end{definition}

The following proof structure is not a proof-net:
\begin{ex}
	\begin{figure}[h]
	\begin{center}
		\includegraphics{figures/pn_ax_tens.pdf}
		\caption{Example of a proof-structure which is not a proof-net}
	\end{center}
	\end{figure}	
\end{ex}

The transformation from proof to proof-structure is then not surjective. One reason is that in proof structures, there is no difference between $\otimes$ and $\parr$.

\subsection{Correctness criterion}

\begin{definition}{Correction graph}
	A correction graph of a proof structure is obtained by removing one of the two upper edges of each $\parr$-node.
\end{definition}

\begin{prop}{Correctnesss criterion (Danos-Regnier)}
	A proof structure is a proof-net if and only if all its correction graphs are connected and acyclic.
\end{prop}
\begin{proof}
\todo[inline]{TODO}
\end{proof}

\subsection{Cut elimination}
	\begin{figure}[h]
	\begin{center}
		\includegraphics{figures/ce_ax.pdf}
		\caption{$ax$ reduction rule.}
		\includegraphics{figures/ce_par_tens.pdf}
		\caption{$\parr - \otimes$ reduction rule.}
		
	\end{center}
	\end{figure}	
	
	Note that the reduced proof-structure is still a proof-net.
	
	\begin{prop}
	Strong normalization, confluence.
	\end{prop}	
	
	\begin{prop}{Sub-formula property}
\todo[inline]{TODO}
	\end{prop}	
	
	\subsection{Exponential connectives}
	
	MLL is not sufficient, computationally speaking, for intuitionistic logic.
	
	\begin{ex}
		$A \rightarrow B \rightarrow A$ is provable in intuitionistic logic, but $A \multimap B \multimap A$ is not provable in MLL.
	\end{ex}	
	
	\begin{definition}{MELL rules:} MLL rules +
	
	$\infer[?_c]{\vdash \Gamma, ?A}{\vdash \Gamma, ?A, ?A}$ (contraction) 
	$\infer[?_w]{\vdash \Gamma, ?A}{\vdash \Gamma}$ (weakening)
	
	$\infer[?_d]{\vdash \Gamma, ?A}{\vdash \Gamma, A}$ (dereliction)
	$\infer[!]{\vdash ?\Gamma, !A}{\vdash ?\Gamma, A}$(promotion)
	\end{definition}
	
	\begin{exo}
		$$\begin{array}{l l l l l}
			LK & \overset{\lnot\lnot}{\longrightarrow} &  LJ & \longrightarrow & MELL \\
			& & \Gamma \vdash A & & !\Gamma ^* \vdash A^* \equiv\ \vdash ?(\Gamma ^ *)^\perp, A^*
		\end{array}$$
		where $(A \rightarrow B)* = ! A^* \multimap B$		
	\end{exo}	

	\begin{figure}[h]
	\begin{center}
		\includegraphics{figures/mell_c.pdf} \quad
		\includegraphics{figures/mell_w.pdf} \quad
		\includegraphics{figures/mell_d.pdf}	\quad
		\includegraphics{figures/mell_p.pdf}	\quad
		\caption{The nodes corresponding to the contraction, weakening, dereliction and promotion rules.}
	\end{center}	
	\end{figure}	
	
	Note that the promotion node is not sufficient in itself to express the promotion rule (\ie the context must be of the form $?\Gamma$).
	The solution is to add boxes (one box per $!$-node):
	\begin{figure}[h]
	\begin{center}
		\includegraphics{figures/box.pdf}
		\caption{Re presentation of a $!$-box.}
	\end{center}	
	\end{figure}
	
	Connectedness of correction graphs is not true anymore. In fact we have:
	
	\begin{prop}
		For any correction graph of a proof-net, we have: \\ $\sharp$ connected components $=1+ \sharp ?-$nodes.
	\end{prop}	
	
	This condition is not sufficient, as shown by the following example:
	
	\begin{ex}
		The correctness graphs of the following example verify the previous property ($2$ connected components), but is not a proof-net.
		\begin{figure}
		\begin{center}
			\includegraphics{figures/connected.pdf}
		\end{center}		
		\end{figure}		
	\end{ex}	
	
	
	\begin{rmk}
		There is no non-trivial overlapping of boxes.
	\end{rmk}	
	
	This allows us to define the depth of a $!$-box by the number of box containing it.
	
	\begin{prop}{Acyclicity criterion}
		In a proof-net, correction graphs are acyclic (where correction graphs are now obtained by removing one of the upper edges of each $\parr$-node and each $?_c$-node.
	\end{prop}	
	
	These properties can be summed up in the following theorem:
	
	\begin{theo}
		$\Pi \vdash_{MELL} \Gamma \Rightarrow$ $\Pi^*$ (and recursively all its boxes) has acyclic correction graphs, and $\sharp cc = \sharp ?_w +1$.
	\end{theo}	
	
	As stated before, the converse is false.
	
	\begin{prop}{Sequentialization}
		If $\mathcal R$ is $?_w$-free with acyclic connected correction graphs, then $\exists \Pi, \Pi* = \mathcal R$.
	\end{prop}	
	
	\begin{exo}
		\begin{multicols}{2}
		\begin{itemize}
			\item $!(A\otimes B)\overset{\dashv}{\vdash} !A \otimes !B$
			\item $!!A \overset{\dashv}{\vdash} !A$
			\item $??A \overset{\dashv}{\vdash} ?A$
			\item $!A \otimes !A\overset{\dashv}{\vdash} !A$
			\item $!A \otimes !B\overset{\dashv}{\vdash} !A$
			\item $?A \otimes ?A\overset{\dashv}{\vdash}?A$
			\item $?A \parr ?A\overset{\dashv}{\vdash}?A$
			\item $!A\overset{\dashv}{\vdash}?A$
			\item $!A\overset{\dashv}{\vdash}A$
		\end{itemize}	
		\end{multicols}
	\end{exo}	
	
	\begin{definition}{Modality}
		Only arity-1 connectives: $\underset{\mu}{\underbrace{!\dots ?\dots ! \dots}}A$
		
		$\mu \sim \nu \Leftrightarrow \forall A, \mu A \overset{\dashv}{\vdash} \nu A$
		
		$\mu \le \nu \Leftrightarrow \forall A, \mu A \vdash \nu A$
	\end{definition}	
	
	\begin{exo}
		Determine $\le$ and $\sim$.
	\end{exo}
	
	\subsection{Cut elimination}
	\begin{rmk}
		A cut from a $!$-node to a $?$-node of the corresponding box is impossible (since the graph of the corresponding box is connected, it would introduce a cycle).
		\end{rmk}		
	The only additional cuts are of the form, and can be reduced as described by the figures~\ref{fig:ce_d}, \ref{fig:ce_w}, \ref{fig:ce_c} and \ref{fig:ce_box}
	
	\begin{figure}[!h]
	\begin{center}	
		\includegraphics[scale=0.6]{figures/ce_d.pdf}
		\caption{Dereliction cut: open the box.}
		\label{fig:ce_d}
	\end{center}	
	\end{figure}	
	
	\begin{figure}[!h]
	\begin{center}
		\includegraphics[scale=0.6]{figures/ce_c.pdf}
		\caption{Contraction cut: duplicate the box.}
		\label{fig:ce_c}
		\end{center}	
	\end{figure}	
	\begin{figure}[!h]
	\begin{center}
		\includegraphics[scale=0.6]{figures/ce_w.pdf}
		\caption{Weakening cut: remove the box.}
		\label{fig:ce_w}
		\end{center}	
	\end{figure}	
	\begin{figure}[!h]
	\begin{center}
		\includegraphics[scale=0.6]{figures/ce_box.pdf}
		\caption{Box-box cut: merge the boxes.}
		\label{fig:ce_box}
	\end{center}	
	\end{figure}	
	
	\begin{exo}
		Check that the correction criterion is preserved through cut reduction.
	\end{exo}	
	
	\begin{prop}{Local confluence}
	\includegraphics[scale=0.5]{figures/confluence.pdf}
	\end{prop}	

		\begin{prop}{Termination}
		\begin{description}
			\item[Weak normalization:] $\forall {\mathcal R}, \exists {\mathcal R_0}$ cut-free  such that ${\mathcal R} \rightarrow^* {\mathcal R_0}$.
			\item[Strong normalization] $\forall {\mathcal R}$, each reduction chain begining by ${\mathcal R}$ is finite.
		\end{description}		
		\end{prop}
		\begin{proof}{(of weak normalization)}
		\todo[inline]{TODO}
		\end{proof}				
\end{document}

	
